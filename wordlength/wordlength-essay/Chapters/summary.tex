\section{Summary}
The Pali language, as every other language, has evolved over time. One of the changes that the language has undergone through the centuries is that words have become increasingly complex and words have merged together to form longer words. We would therefore expect that a statistical analysis of average word-length could be an indication of relative "age" of the texts in question. 

In this research I analyse the average word-length of each text and each book in the Pali canon to find trends in the canon and test the above hypothesis; to show that average word-length can be used as an indicator to show the development of the texts over time.\\

Average word-length in itself remains only an indication of "lateness" of texts, among other indications like the existence and quality of parallels, but nevertheless can prove a valueble resource in determining the development of the texts in the Pali canon and ultimately what the Buddha taught.

The analysis of word-length is more reliable on larger texts, or whole collections and books, than it is on individual texts; if texts are heavily abbreviated or consist only of a few lines, the word-length can give a value that is not representative of the collection as a whole and can greatly vary.
